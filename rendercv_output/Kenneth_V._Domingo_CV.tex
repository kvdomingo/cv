\documentclass[10pt, a4paper]{article}

% Packages:
\usepackage[
    ignoreheadfoot, % set margins without considering header and footer
    top=2 cm, % seperation between body and page edge from the top
    bottom=2 cm, % seperation between body and page edge from the bottom
    left=2 cm, % seperation between body and page edge from the left
    right=2 cm, % seperation between body and page edge from the right
    footskip=1.0 cm, % seperation between body and footer
    % showframe % for debugging 
]{geometry} % for adjusting page geometry
\usepackage{titlesec} % for customizing section titles
\usepackage{tabularx} % for making tables with fixed width columns
\usepackage{array} % tabularx requires this
\usepackage[dvipsnames]{xcolor} % for coloring text
\definecolor{primaryColor}{RGB}{67, 56, 202} % define primary color
\usepackage{enumitem} % for customizing lists
\usepackage{fontawesome5} % for using icons
\usepackage{amsmath} % for math
\usepackage[
    pdftitle={Kenneth V. Domingo's CV},
    pdfauthor={Kenneth V. Domingo},
    pdfcreator={LaTeX with RenderCV},
    colorlinks=true,
    urlcolor=primaryColor
]{hyperref} % for links, metadata and bookmarks
\usepackage[pscoord]{eso-pic} % for floating text on the page
\usepackage{calc} % for calculating lengths
\usepackage{bookmark} % for bookmarks
\usepackage{lastpage} % for getting the total number of pages
\usepackage{changepage} % for one column entries (adjustwidth environment)
\usepackage{paracol} % for two and three column entries
\usepackage{ifthen} % for conditional statements
\usepackage{needspace} % for avoiding page brake right after the section title
\usepackage{iftex} % check if engine is pdflatex, xetex or luatex

% Ensure that generate pdf is machine readable/ATS parsable:
\ifPDFTeX
    \input{glyphtounicode}
    \pdfgentounicode=1
    % \usepackage[T1]{fontenc} % this breaks sb2nov
    \usepackage[utf8]{inputenc}
    \usepackage{lmodern}
\fi



% Some settings:
\AtBeginEnvironment{adjustwidth}{\partopsep0pt} % remove space before adjustwidth environment
\pagestyle{empty} % no header or footer
\setcounter{secnumdepth}{0} % no section numbering
\setlength{\parindent}{0pt} % no indentation
\setlength{\topskip}{0pt} % no top skip
\setlength{\columnsep}{0cm} % set column seperation
\makeatletter
\let\ps@customFooterStyle\ps@plain % Copy the plain style to customFooterStyle
\patchcmd{\ps@customFooterStyle}{\thepage}{
    \color{gray}\textit{\small Kenneth V. Domingo - Page \thepage{} of \pageref*{LastPage}}
}{}{} % replace number by desired string
\makeatother
\pagestyle{customFooterStyle}

\titleformat{\section}{\needspace{4\baselineskip}\bfseries\large}{}{0pt}{}[\vspace{1pt}\titlerule]

\titlespacing{\section}{
    % left space:
    -1pt
}{
    % top space:
    0.3 cm
}{
    % bottom space:
    0.2 cm
} % section title spacing

\renewcommand\labelitemi{$\circ$} % custom bullet points
\newenvironment{highlights}{
    \begin{itemize}[
        topsep=0.10 cm,
        parsep=0.10 cm,
        partopsep=0pt,
        itemsep=0pt,
        leftmargin=0.4 cm + 10pt
    ]
}{
    \end{itemize}
} % new environment for highlights

\newenvironment{highlightsforbulletentries}{
    \begin{itemize}[
        topsep=0.10 cm,
        parsep=0.10 cm,
        partopsep=0pt,
        itemsep=0pt,
        leftmargin=10pt
    ]
}{
    \end{itemize}
} % new environment for highlights for bullet entries


\newenvironment{onecolentry}{
    \begin{adjustwidth}{
        0.2 cm + 0.00001 cm
    }{
        0.2 cm + 0.00001 cm
    }
}{
    \end{adjustwidth}
} % new environment for one column entries

\newenvironment{twocolentry}[2][]{
    \onecolentry
    \def\secondColumn{#2}
    \setcolumnwidth{\fill, 4.5 cm}
    \begin{paracol}{2}
}{
    \switchcolumn \raggedleft \secondColumn
    \end{paracol}
    \endonecolentry
} % new environment for two column entries

\newenvironment{header}{
    \setlength{\topsep}{0pt}\par\kern\topsep\centering\linespread{1.5}
}{
    \par\kern\topsep
} % new environment for the header

\newcommand{\placelastupdatedtext}{% \placetextbox{<horizontal pos>}{<vertical pos>}{<stuff>}
  \AddToShipoutPictureFG*{% Add <stuff> to current page foreground
    \put(
        \LenToUnit{\paperwidth-2 cm-0.2 cm+0.05cm},
        \LenToUnit{\paperheight-1.0 cm}
    ){\vtop{{\null}\makebox[0pt][c]{
        \small\color{gray}\textit{}\hspace{\widthof{}}
    }}}%
  }%
}%

% save the original href command in a new command:
\let\hrefWithoutArrow\href

% new command for external links:
\renewcommand{\href}[2]{\hrefWithoutArrow{#1}{\ifthenelse{\equal{#2}{}}{ }{#2 }\raisebox{.15ex}{\footnotesize \faExternalLink*}}}


\begin{document}
    \newcommand{\AND}{\unskip
        \cleaders\copy\ANDbox\hskip\wd\ANDbox
        \ignorespaces
    }
    \newsavebox\ANDbox
    \sbox\ANDbox{}

    \placelastupdatedtext
    \begin{header}
        \textbf{\fontsize{30 pt}{30 pt}\selectfont Kenneth V. Domingo}

        \vspace{0.3 cm}

        \normalsize
        \mbox{{\color{black}\footnotesize\faMapMarker*}\hspace*{0.13cm}Bulacan, Philippines}%
        \kern 0.25 cm%
        \AND%
        \kern 0.25 cm%
        \mbox{\hrefWithoutArrow{mailto:hello@kvd.studio}{\color{black}{\footnotesize\faEnvelope[regular]}\hspace*{0.13cm}hello@kvd.studio}}%
        \kern 0.25 cm%
        \AND%
        \kern 0.25 cm%
        \mbox{\hrefWithoutArrow{https://kvd.studio/}{\color{black}{\footnotesize\faLink}\hspace*{0.13cm}kvd.studio}}%
        \kern 0.25 cm%
        \AND%
        \kern 0.25 cm%
        \mbox{\hrefWithoutArrow{https://linkedin.com/in/kvdomingo}{\color{black}{\footnotesize\faLinkedinIn}\hspace*{0.13cm}kvdomingo}}%
        \kern 0.25 cm%
        \AND%
        \kern 0.25 cm%
        \mbox{\hrefWithoutArrow{https://github.com/kvdomingo}{\color{black}{\footnotesize\faGithub}\hspace*{0.13cm}kvdomingo}}%
    \end{header}

    \vspace{0.3 cm - 0.3 cm}


    \section{Work Experience}



        
        \begin{twocolentry}{
        \textit{Taguig City, PH}    
            
        \textit{Jan 2023 – Oct 2024}}
            \textbf{Software Engineer}
            
            \textit{Thinking Machines Data Science}
        \end{twocolentry}

        \vspace{0.10 cm}
        \begin{onecolentry}
            \begin{highlights}
                \item Led a cross-functional team of 7 engineers to architect and develop the UNICEF  \href{https://sync.giga.global}{Giga DataOps Platform}, a fully  \href{https://github.com/unicef/giga-dagster}{open-source} master data management platform built in coordination with \href{https://giga.global}{Giga}, a UNICEF-ITU joint initiative whose mission is to connect all schools to the internet by 2030. V1.0 of the Platform was launched in September 2024 and  was presented at the  \href{https://osseu2024.sched.com/event/1iSnq/global-school-connectivity-dataops-platform-shilpa-arora-unicef}{Open Source Summit}  in Vienna later that month. At launch, the Platform managed data for over 2.1M school records across 141  countries, with near-real-time connectivity data for 93k schools updating every few minutes. The Platform  was also being leveraged by 2 downstream applications with 5k monthly users.

                \item Developed a data analytics platform for a foreign agricultural company: optimized the data lake setup for near-realtime updates of logistic movements and validation of the delivery of burnt produce. Daily tasks involved orchestration and scheduling of ELT tasks and SQL queries via Dagster, and automating replication of hybrid cloud deployment environments via Terraform.

                \item Developed \href{https://thinkingmachin.es/llm-solutions/}{GenAI applications} for several enterprise clients in  the financial sector, leveraging Azure OpenAI. Daily tasks involved designing responsive chat interfaces, and image processing to remove non-text elements in financial documents in order to prepare them for vector embedding for use in downstream RAG applications.

                \item Developed the \href{https://sdg-tool.goldstandard.org/}{SDG Impact Tool} for  \href{https://goldstandard.org}{Gold Standard} to digitize their process of reporting, validating, verifying, and tracking the contribution of project activities to the UN Sustainable Development Goals. Daily tasks involved backend development in Django, frontend development in React, and automating infrastructure provisioning across multiple Google Cloud deployment environments using Terraform.

                \item Developed an internal geospatial analytics platform for a local telecommunications company: involved in migrating user interface components to a modern tech stack and increasing test coverage.

            \end{highlights}
        \end{onecolentry}


        \vspace{0.2 cm}

        \begin{twocolentry}{
        \textit{Remote}    
            
        \textit{Nov 2022 – Jan 2023}}
            \textbf{Software Developer (Freelance)}
            
            \textit{Stevn Books}
        \end{twocolentry}

        \vspace{0.10 cm}
        \begin{onecolentry}
            \begin{highlights}
                \item Developed the API and infrastructure for an inventory management and profit calculator system deployed on DigitalOcean: integrated with PayMongo to process customer payments and AWS Cognito as a B2C identity provider.

            \end{highlights}
        \end{onecolentry}


        \vspace{0.2 cm}

        \begin{twocolentry}{
        \textit{Taguig City, PH}    
            
        \textit{Nov 2020 – Nov 2022}}
            \textbf{Software Engineer}
            
            \textit{Demand Science (Cobena Business Analytics and Strategy)}
        \end{twocolentry}

        \vspace{0.10 cm}
        \begin{onecolentry}
            \begin{highlights}
                \item Developed a unified frontend enabling users to experiment with deployed ML models being developed in-house.

                \item Developed the frontend for an internal data lakehouse platform allowing users to perform queries on access-controlled data, apply filters, save filters, and export their search results.

                \item Guided the adoption of MLOps principles within the AI/ML team.
                \item Spearheaded the company-wide adoption of modern frontend tooling using Figma and TypeScript.
                \item Developed a supply chain performance survey tool for UK-based marketing consultancy \href{https://ideasandaction.com/}{Ideas and Action}: led backend development with Django, Redis, and PostgreSQL, deployed to an AWS environment.

                \item Developed Gateway, a B2B SaaS geospatial analytics platform: led weekly code reviews to maintain quality standards and optimize the team’s delivery.

                \item Developed internal tooling for an FMCG client: initiated the migration from manual testing to automated API testing using Python scripts, reducing testing time from 2 days to 30 minutes.

            \end{highlights}
        \end{onecolentry}


        \vspace{0.2 cm}

        \begin{twocolentry}{
        \textit{Quezon City, PH}    
            
        \textit{Mar 2020 – May 2020}}
            \textbf{Backend Developer (Volunteer)}
            
            \textit{DetectPH}
        \end{twocolentry}

        \vspace{0.10 cm}
        \begin{onecolentry}
            \begin{highlights}
                \item Developed the API for a COVID-19 contact-tracing mobile application written with Express and MongoDB.
            \end{highlights}
        \end{onecolentry}


        \vspace{0.2 cm}

        \begin{twocolentry}{
        \textit{Quezon City, PH}    
            
        \textit{Jan 2020 – Feb 2020}}
            \textbf{Bioinformatics Intern}
            
            \textit{Philippine Genome Center}
        \end{twocolentry}

        \vspace{0.10 cm}
        \begin{onecolentry}
            \begin{highlights}
                \item Underwent training in bioinformatics; developed a command-line tool and web application for designing  primers for site-directed mutagenesis.

            \end{highlights}
        \end{onecolentry}



    
    \section{Certifications}



        
        \begin{twocolentry}{
            
            
        \textit{May 2024}}
            \textbf{Azure AI Engineer Associate}
        \end{twocolentry}

        \vspace{0.10 cm}
        \begin{onecolentry}
            \begin{highlights}
                \item \href{https://learn.microsoft.com/api/credentials/share/en-us/KennethDomingo-4666/970CBB2EEB4ACD4A?sharingId=791A05E133CA70A}{Microsoft}
            \end{highlights}
        \end{onecolentry}


        \vspace{0.2 cm}

        \begin{twocolentry}{
            
            
        \textit{July 2021}}
            \textbf{Professional Cloud Architect}
        \end{twocolentry}

        \vspace{0.10 cm}
        \begin{onecolentry}
            \begin{highlights}
                \item \href{https://www.credential.net/0300e26a-fcdc-40db-a4a0-689fad65ac9b\#gs.755e2e}{Google Cloud}
            \end{highlights}
        \end{onecolentry}



    
    \section{Education}



        
        \begin{twocolentry}{
            
            
        \textit{Aug 2015 – July 2020}}
            \textbf{University of the Philippines-Diliman}

            \textit{B.S. in Applied Physics (Major in Instrumentation)}
        \end{twocolentry}

        \vspace{0.10 cm}
        \begin{onecolentry}
            \begin{highlights}
                \item \textbf{Thesis:} Compressive sensing: Applications from 1-D to $N$-D
            \end{highlights}
        \end{onecolentry}



    
    \section{Publications}



        
        \begin{samepage}
            \begin{twocolentry}{
                Sept 2020
            }
                \textbf{Compressively sampled speech: How good is the recovery?}

                \vspace{0.10 cm}

                \mbox{K. V. Domingo}, \mbox{M. N. Soriano}
            \end{twocolentry}


            \vspace{0.10 cm}

            \begin{onecolentry}
        Proceedings of the Samahang Pisika ng Pilipinas 38 \href{https://paperview.spp-online.org/proceedings/article/view/SPP-2020-4C-04}{SPP-2020-4C-04}
            \end{onecolentry}
        \end{samepage}

        \vspace{0.2 cm}

        \begin{samepage}
            \begin{twocolentry}{
                May 2019
            }
                \textbf{Frequency domain reconstruction of stochastically sampled signals based on compressive sensing}

                \vspace{0.10 cm}

                \mbox{K. V. Domingo}, \mbox{M. N. Soriano}
            \end{twocolentry}


            \vspace{0.10 cm}

            \begin{onecolentry}
        Proceedings of the Samahang Pisika ng Pilipinas 37 \href{https://paperview.spp-online.org/proceedings/article/view/SPP-2019-PB-38}{SPP-2019-PB-38}
            \end{onecolentry}
        \end{samepage}


    
    \section{Projects}



        
        \begin{twocolentry}{
            
            
        \textit{\href{https://banyuhay.kvd.studio}{Banyuhay}}}
            \textbf{Banyuhay}
        \end{twocolentry}

        \vspace{0.10 cm}
        \begin{onecolentry}
            \begin{highlights}
                \item Developed a geospatial web app to allow users to locate nearby restrooms with bidets.

                \item \textbf{Tools Used}: Python, FastAPI, TypeScript, SvelteKit, OpenStreetMaps
            \end{highlights}
        \end{onecolentry}


        \vspace{0.2 cm}

        \begin{twocolentry}{
            
            
        \textit{\href{https://sync.giga.global}{Giga Sync}}}
            \textbf{Giga DataOps Platform}
        \end{twocolentry}

        \vspace{0.10 cm}
        \begin{onecolentry}
            \begin{highlights}
                \item An open-source master data management platform developed with Giga, a UNICEF x ITU initiative to connect all  schools around the world to the internet by 2030.

                \item \textbf{Tools Used}: Python, FastAPI, TypeScript, React, Dagster, Docker, Azure Kubernetes Service, Helm, Azure DevOps, Apache Spark, Apache Superset, Delta Lake, Datahub, Azure Data Lake Storage Gen2, Trino,  Apache Hive Metastore, Carbon Design System, TailwindCSS, PostgreSQL, Redis, Prometheus, Grafana, Sentry

            \end{highlights}
        \end{onecolentry}


        \vspace{0.2 cm}

        \begin{twocolentry}{
            
            
        \textit{\href{https://gpt.kvd.studio}{KenGPT}}}
            \textbf{KenGPT}
        \end{twocolentry}

        \vspace{0.10 cm}
        \begin{onecolentry}
            \begin{highlights}
                \item Yet another ChatGPT clone.
                \item \textbf{Tools Used}: TypeScript, React, Next.js, OpenAI, TailwindCSS
            \end{highlights}
        \end{onecolentry}


        \vspace{0.2 cm}

        \begin{twocolentry}{
            
            
        \textit{\href{https://lsfm.kvd.studio}{LSFM}}}
            \textbf{LSFM}
        \end{twocolentry}

        \vspace{0.10 cm}
        \begin{onecolentry}
            \begin{highlights}
                \item A clone of the LE SSERAFIM Digital Souvenir system.
                \item \textbf{Tools Used}: TypeScript, React, TailwindCSS, Shadcn UI, WebAssembly, ffmpeg
            \end{highlights}
        \end{onecolentry}


        \vspace{0.2 cm}

        \begin{twocolentry}{
            
            
        \textit{\href{https://chaebot.kvd.studio}{Chaebot}}}
            \textbf{Chaebot}
        \end{twocolentry}

        \vspace{0.10 cm}
        \begin{onecolentry}
            \begin{highlights}
                \item A Discord bot that keeps track of upcoming K-pop releases, and crossposts Twitter and VLIVE media to  K-pop-oriented Discord guilds.

                \item \textbf{Tools Used}: Python, Flask, Discord.py, JavaScript, React, Twitter API, Material UI, PostgreSQL
            \end{highlights}
        \end{onecolentry}


        \vspace{0.2 cm}

        \begin{twocolentry}{
            
            
        \textit{\href{https://primerdriver.kvd.studio}{PrimerDriver}}}
            \textbf{PrimerDriver}
        \end{twocolentry}

        \vspace{0.10 cm}
        \begin{onecolentry}
            \begin{highlights}
                \item CLI and web application for automation of the design of mutagenic PCR primers for site-directed mutagenesis.
                \item \textbf{Tools Used}: Python, Flask, JavaScript, React, BioPython, Click, Sphinx, Material Design Bootstrap
            \end{highlights}
        \end{onecolentry}



    
    \section{Technologies}



        
        \begin{onecolentry}
            \textbf{Backend:} Python (Django, FastAPI, Flask), Go (Fiber), Node.js (Express, Hono)
        \end{onecolentry}

        \vspace{0.2 cm}

        \begin{onecolentry}
            \textbf{Frontend:} TypeScript (React, Next.js, Svelte), Static (Astro, HTML, CSS)
        \end{onecolentry}

        \vspace{0.2 cm}

        \begin{onecolentry}
            \textbf{Databases:} Relational (PostgreSQL, SQLite), NoSQL (Redis, Firestore, MongoDB)
        \end{onecolentry}

        \vspace{0.2 cm}

        \begin{onecolentry}
            \textbf{Cloud Platforms:} Google Cloud Platform, AWS, Azure, Vercel, Firebase
        \end{onecolentry}

        \vspace{0.2 cm}

        \begin{onecolentry}
            \textbf{DevSecOps:} Git, Docker, Kubernetes, Helm, Snyk, Ansible, ArgoCD, GitHub, GitHub Actions, GitHub Advanced Security, Azure DevOps

        \end{onecolentry}

        \vspace{0.2 cm}

        \begin{onecolentry}
            \textbf{DataOps:} Dagster, Google BigQuery, Apache Spark
        \end{onecolentry}

        \vspace{0.2 cm}

        \begin{onecolentry}
            \textbf{Visualization \& Dashboarding:} Apache Superset, Metabase
        \end{onecolentry}

        \vspace{0.2 cm}

        \begin{onecolentry}
            \textbf{Infrastructure:} Linux, Terraform, OpenTofu, Prometheus, Grafana
        \end{onecolentry}


    

\end{document}